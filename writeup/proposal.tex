% Options for packages loaded elsewhere
\PassOptionsToPackage{unicode}{hyperref}
\PassOptionsToPackage{hyphens}{url}
%
\documentclass[
  12pt,
  a4paper,
]{article}
\usepackage{amsmath,amssymb}
\usepackage{iftex}
\ifPDFTeX
  \usepackage[T1]{fontenc}
  \usepackage[utf8]{inputenc}
  \usepackage{textcomp} % provide euro and other symbols
\else % if luatex or xetex
  \usepackage{unicode-math} % this also loads fontspec
  \defaultfontfeatures{Scale=MatchLowercase}
  \defaultfontfeatures[\rmfamily]{Ligatures=TeX,Scale=1}
\fi
\usepackage{lmodern}
\ifPDFTeX\else
  % xetex/luatex font selection
\fi
% Use upquote if available, for straight quotes in verbatim environments
\IfFileExists{upquote.sty}{\usepackage{upquote}}{}
\IfFileExists{microtype.sty}{% use microtype if available
  \usepackage[]{microtype}
  \UseMicrotypeSet[protrusion]{basicmath} % disable protrusion for tt fonts
}{}
\usepackage{xcolor}
\usepackage[margin=1in]{geometry}
\usepackage{longtable,booktabs,array}
\usepackage{calc} % for calculating minipage widths
% Correct order of tables after \paragraph or \subparagraph
\usepackage{etoolbox}
\makeatletter
\patchcmd\longtable{\par}{\if@noskipsec\mbox{}\fi\par}{}{}
\makeatother
% Allow footnotes in longtable head/foot
\IfFileExists{footnotehyper.sty}{\usepackage{footnotehyper}}{\usepackage{footnote}}
\makesavenoteenv{longtable}
\usepackage{graphicx}
\makeatletter
\def\maxwidth{\ifdim\Gin@nat@width>\linewidth\linewidth\else\Gin@nat@width\fi}
\def\maxheight{\ifdim\Gin@nat@height>\textheight\textheight\else\Gin@nat@height\fi}
\makeatother
% Scale images if necessary, so that they will not overflow the page
% margins by default, and it is still possible to overwrite the defaults
% using explicit options in \includegraphics[width, height, ...]{}
\setkeys{Gin}{width=\maxwidth,height=\maxheight,keepaspectratio}
% Set default figure placement to htbp
\makeatletter
\def\fps@figure{htbp}
\makeatother
\setlength{\emergencystretch}{3em} % prevent overfull lines
\providecommand{\tightlist}{%
  \setlength{\itemsep}{0pt}\setlength{\parskip}{0pt}}
\setcounter{secnumdepth}{5}
% definitions for citeproc citations
\NewDocumentCommand\citeproctext{}{}
\NewDocumentCommand\citeproc{mm}{%
  \begingroup\def\citeproctext{#2}\cite{#1}\endgroup}
\makeatletter
 % allow citations to break across lines
 \let\@cite@ofmt\@firstofone
 % avoid brackets around text for \cite:
 \def\@biblabel#1{}
 \def\@cite#1#2{{#1\if@tempswa , #2\fi}}
\makeatother
\newlength{\cslhangindent}
\setlength{\cslhangindent}{1.5em}
\newlength{\csllabelwidth}
\setlength{\csllabelwidth}{3em}
\newenvironment{CSLReferences}[2] % #1 hanging-indent, #2 entry-spacing
 {\begin{list}{}{%
  \setlength{\itemindent}{0pt}
  \setlength{\leftmargin}{0pt}
  \setlength{\parsep}{0pt}
  % turn on hanging indent if param 1 is 1
  \ifodd #1
   \setlength{\leftmargin}{\cslhangindent}
   \setlength{\itemindent}{-1\cslhangindent}
  \fi
  % set entry spacing
  \setlength{\itemsep}{#2\baselineskip}}}
 {\end{list}}
\usepackage{calc}
\newcommand{\CSLBlock}[1]{\hfill\break\parbox[t]{\linewidth}{\strut\ignorespaces#1\strut}}
\newcommand{\CSLLeftMargin}[1]{\parbox[t]{\csllabelwidth}{\strut#1\strut}}
\newcommand{\CSLRightInline}[1]{\parbox[t]{\linewidth - \csllabelwidth}{\strut#1\strut}}
\newcommand{\CSLIndent}[1]{\hspace{\cslhangindent}#1}
\usepackage{amsmath}
\numberwithin{equation}{section}
\usepackage{setspace}\onehalfspacing
\usepackage{pdflscape}
\newcommand{\blandscape}{\begin{landscape}}
\newcommand{\elandscape}{\end{landscape}}
\usepackage{indentfirst}
\usepackage{xparse}
\usepackage{algorithm}
\usepackage{algpseudocode}
\usepackage{float}
\renewcommand{\ttfamily}{\rmfamily}
\ifLuaTeX
  \usepackage{selnolig}  % disable illegal ligatures
\fi
\usepackage{bookmark}
\IfFileExists{xurl.sty}{\usepackage{xurl}}{} % add URL line breaks if available
\urlstyle{same}
\hypersetup{
  hidelinks,
  pdfcreator={LaTeX via pandoc}}

\author{}
\date{\vspace{-2.5em}}

\begin{document}

\title{\vspace{20mm}TITLE HERE}
\maketitle
\thispagestyle{empty} 
\vspace*{95px}
\begin{center}
A Thesis Proposal Presented to\\
The Faculty of the School of Statistics\\
univ\\
\vspace*{100px}
In Partial Fulfillment\\
of the Requirements for the Degree of\\
Master of Science in Statistics\\
1st Semester A.Y. 2025-2026

\vspace*{100px}
by\\
Shaine Rosewel Matala
\end{center}

\newpage

\newpage

\tableofcontents

\newpage

\section{Introduction}\label{introduction}

\section{Related Literature}\label{related-literature}

\subsection{Joint confidence region for an overall ranking}\label{joint-confidence-region-for-an-overall-ranking}

\(\Bigcap\)

\subsection{Joint confidence region construction}\label{joint-confidence-region-construction}

\subsection{\texorpdfstring{\(T_1, T_2, T_3\)}{T\_1, T\_2, T\_3}}\label{t_1-t_2-t_3}

\section{Methodology}\label{methodology}

This section introduces the proposed methodologies to obtain confidence regions for the unknown overall true ranking. The following cases are tackled: case when items ranked are assumed to have zero and nonzero correlation. Both approaches are based on parametric bootstrap. Sections \ref{sec:parametricbs} and \ref{sec:nonrankbased} discuss the algorithms for the cases mentioned. Section \ref{sec:evaluation} shows the algorithms used to assess the performance of the proposed approaches. This makes use of coverage and metrics to measure the tightness of the estimated confidence regions.

For sections \ref{sec:parametricbs} and \ref{sec:nonrankbased}, let \(\theta_1, \theta_2, \dots, \theta_K\) be the true parameter values and \(\hat \theta_1, \hat \theta_2, \dots, \hat \theta_K\) be the corresponding estimates.

\subsection{\texorpdfstring{Joint confidence intervals for \(\theta_1, \dots, \theta_K\) by using Parametric Bootstrap}{Joint confidence intervals for \textbackslash theta\_1, \textbackslash dots, \textbackslash theta\_K by using Parametric Bootstrap}}\label{sec:parametricbs}

The rank-based parametric bootstrap approach assumes \(\hat \theta_1, \hat \theta_2, \dots, \hat \theta_K\) to be independent but not identically distributed estimates, where \(\hat{\theta}_k \sim N \left(\theta_k, \sigma_k^2 \right)\), \(k = 1, 2, \dots, K\). \(\sigma^2_k\) is assumed known. Denote the corresponding ordered values by \(\hat \theta_{(1)}, \hat \theta_{(2)}, \dots, \hat \theta_{(K)}\).

\begin{algorithm}[H]
    \caption{Computation of Joint Confidence Region using Parametric Bootstrap} 
    \label{alg:parametricbs_ci}
    \begin{algorithmic}[1]
        \For {$b = 1, 2, \dots, B$}
                \State Generate $\hat\theta^*_{bk} \sim N \left( \hat\theta_k, \sigma_k^2 \right)$, $k = 1, 2, \dots, K$ and let $\hat{\theta}_{b(1)}, \hat{\theta}_{b(2)}, \dots, \hat{\theta}_{b(K)}$ be the corresponding ordered values
            \Statex \begin{minipage}{\linewidth}
          \centering
          \begin{tabular}{|c|c|c|c|c|}
            \hline
             & $k = 1$ & $k = 2$ & $\dots$ & $k = K$ \\
            \hline
            $b = 1$ & $\hat{\theta}^*_{1(1)}$ & $\hat{\theta}^*_{1(2)}$ & \dots & $\hat{\theta}^*_{1(K)}$ \\
            \hline
            $b = 2$ & $\hat{\theta}^*_{2(1)}$ & $\hat{\theta}^*_{2(2)}$ & \dots & $\hat{\theta}^*_{2(K)}$ \\
            \hline
            $\vdots$ & \vdots & \vdots & \dots & \vdots\\
            \hline
            $b = B$ & $\hat{\theta}^*_{B(1)}$ & $\hat{\theta}^*_{B(2)}$ & \dots & $\hat{\theta}^*_{B(K)}$ \\
            \hline
          \end{tabular}
        \end{minipage}
        \State Compute 
        \Statex \begin{minipage}{\linewidth}
        \centering
$\hat\sigma^*_{b(k)} = \sqrt{\text{kth ordered value among} \ \left\{ \hat{\theta}^{*2}_{b1} + \sigma_1^2, \hat{\theta}^{*2}_{b2} + \sigma_2^2, \dots, \hat{\theta}^{*2}_{bK} + \sigma_K^2 \right\} - \hat {\theta}^{*2}_{(k)}}$
        \end{minipage}
                \State Compute $t^*_b = \underset{1 \leq k \leq K}{\max} \Bigg| \frac{\hat\theta^*_{b(k)} - \hat\theta^*_{k}}{\sigma^*_{b(k)}} \Bigg|$
        \EndFor
        \State Compute the $\left(1-\alpha\right)$-sample quantile of $t^*_1, t^*_2, \dots, t^*_B$, call this $\hat{t}$.
        \State The joint confidence region of $\theta_{(1)}, \theta_{(2)}, \dots, \theta_{(K)}$ is given by 
        \Statex \begin{minipage}{\linewidth}
    \centering
$\mathfrak{R} = \left[ \hat\theta_{(1)} \pm \hat t \times \hat\sigma_{(1)}  \right] \times \left[ \hat\theta_{(2)} \pm \hat t \times \hat\sigma_{(2)}  \right] \times \dots \times \left[ \hat\theta_{(K)} \pm \hat t \times \hat\sigma_{(K)}  \right]$
    \end{minipage}
         where $\hat \sigma_{(k)}$ is computed as
        \Statex \begin{minipage}{\linewidth}
    \centering
$\hat\sigma_{(k)} = \sqrt{\text{kth ordered value among} \ \left\{ \hat{\theta}^{2}_{1} + \sigma_1^2, \hat{\theta}^{2}_{2} + \sigma_2^2, \dots, \hat{\theta}^{2}_{K} + \sigma_K^2 \right\} - \hat {\theta}^{2}_{(k)}}$
\end{minipage}
    \end{algorithmic} 
\end{algorithm}

\subsection{\texorpdfstring{Joint confidence intervals for \(\theta_1, \dots, \theta_K\) by using Nonrank-based method}{Joint confidence intervals for \textbackslash theta\_1, \textbackslash dots, \textbackslash theta\_K by using Nonrank-based method}}\label{sec:nonrankbased}

The nonrank-based method assumes that \(\boldsymbol{\hat\theta} = \left(\hat\theta_1, \hat\theta_2, \dots, \hat\theta_K\right) \sim N \left( \boldsymbol{\theta}, \boldsymbol{\Sigma}\right)\). It accounts for potential correlation among items being ranked. For this case, an exchangeable correlation, \(\boldsymbol{\rho}\) (See Equation \ref{eq:equicorrelation}.), is assumed and used in the calculation of the variance covariance matrix (See Equation \ref{eq:sigma_matrix}.).

\begin{equation}
  \boldsymbol{\rho} = \left( 1-\rho \right) \mathbf{I}_K + \rho \boldsymbol{1}_K \boldsymbol{1}'_K
  \label{eq:equicorrelation}
\end{equation}

\begin{equation}
  \boldsymbol{\Sigma} = \boldsymbol{\Delta}^{1/2} \boldsymbol{\rho} \boldsymbol{\Delta}^{1/2}
  \label{eq:sigma_matrix}
\end{equation}

where \(\boldsymbol{\Delta} = \text{diag} \left\{ \sigma^2_1, \sigma^2_2, \dots, \sigma^2_K \right\}\), with known \(\sigma_k\)'s and \(\rho\) is studied for \(0.1, 0.5, 0.9\).

\begin{algorithm}[H]
    \caption{Computation of Joint Confidence Region using Nonrank-based Method} 
    \label{alg:nonrank_ci}
    Let the data consist of $\hat \theta_1, \dots, \hat \theta_K$ and suppose $\boldsymbol{\Sigma}$ is known
    \begin{algorithmic}[1]
        \For {$b = 1, 2, \dots, B$}
                \State Generate $\boldsymbol{\hat\theta}^*_b \sim N_K \left( \boldsymbol{\hat\theta}, \boldsymbol{\Sigma}\right)$ and write $\boldsymbol{\hat\theta}^*_b = \left( \hat\theta^*_{b1}, \hat\theta^*_{b2}, \dots, \hat\theta^*_{bK} \right)' $
                \State Compute $t^*_b = \underset{1 \leq k \leq K}{\max} \Bigg| \frac{\hat\theta^*_{bk} - \hat\theta^*_{k}}{\sigma_k} \Bigg|$
        \EndFor
        \State Compute the $\left(1-\alpha\right)$-sample quantile of $t^*_1, t^*_2, \dots, t^*_B$, call this $\hat{t}$.
        \State The joint confidence region of $\theta_1, \theta_2, \dots, \theta_K$ is given by 
        \Statex \begin{minipage}{\linewidth}
    \centering
$\mathfrak{R} = \left[ \hat\theta_1 \pm \hat t \times \sigma_1  \right] \times \left[ \hat\theta_2 \pm \hat t \times \sigma_2  \right] \times \dots \times \left[ \hat\theta_K \pm \hat t \times \sigma_K  \right]$
    \end{minipage}
    \end{algorithmic} 
\end{algorithm}

\subsection{Evaluation}\label{sec:evaluation}

Algorithm \ref{alg:parametricbs_cov} is used to calculate the coverage which is defined as the proportion of times that the true parameter values fall within the confidence interval for all \(K\) simultaneously. Ideally, this should be equal to \(0.90\) since \(\alpha = 0.1\). It also calculates the average \(T_1, T_2,\) and \(T_3\). Higher values of \(T_1\) and \(T_2\) indicate wider confidence intervals and are therefore less desirable, whereas higher values of \(T_3\) are preferable.

\begin{algorithm}[H]
    \caption{Computation of Coverage Probability for Parametric Bootstrap} 
    \label{alg:parametricbs_cov}
    For given values of $\theta_1, \theta_2, \dots, \theta_K$ and thus $\theta_{(1)}, \theta_{(2)}, \dots, \theta_{(K)}$
    \begin{algorithmic}[1] % Start algorithmic block
            \For {$\text{replications} = 1, 2, \dots, 5000$}
            \State Generate $\hat\theta_k \sim N(\theta_k, \sigma^2_k)$, for $k = 1, 2, \dots, K$
            \State Compute the rectangular confidence region $\mathfrak{R}$ using Algorithm \ref{alg:parametricbs_ci}.
            \State Check if $\left( \theta_{(1)}, \theta_{(2)}, \dots, \theta_{(K)}\right) \in \mathfrak{R}$ and compute 
            \Statex \begin{minipage}{\linewidth}
        \centering
            $T_1 = \frac{1}{K} \sum^K_{k=1} \Big | \Lambda_{Ok} \Big|$\\
            $T_2 = \prod^K_{k=1} \Big | \Lambda_{Ok} \Big|$\\
            $T_3 = 1 - \frac{K + \sum^K_{k=1} \big | \Lambda_{Ok} \big|}{K^2}$\\
            \end{minipage}
        \EndFor
    \State Compute the proportion of times that the condition in step 4 is satisfied and the average of $T_1, T_2$, and $T_3$.
    \end{algorithmic} % End algorithmic block
\end{algorithm}

Algorithm \ref{alg:nonrank_cov} is similar to Algorithm \ref{alg:parametricbs_cov} but computes for the coverage and average \(T_1, T_2,\) and \(T_3\) for the nonrank-based method.

\begin{algorithm}[H]
    \caption{Computation of Coverage Probability for Nonrank-based Method} 
    \label{alg:nonrank_cov}
    For given values of $\theta_1, \theta_2, \dots, \theta_K$ and $\boldsymbol{\Sigma}$
    \begin{algorithmic}[1] % Start algorithmic block
            \For {$\text{replications} = 1, 2, \dots, 5000$}
            \State Generate $\boldsymbol{\hat\theta} \sim N_K(\boldsymbol{\theta}, \boldsymbol{\Sigma})$
            \State Compute the rectangular confidence region $\mathfrak{R}$ using Algorithm \ref{alg:nonrank_ci}.
            \State Check if $\left( \theta_1, \theta_2, \dots, \theta_K\right) \in \mathfrak{R}$ and compute $T_1, T_2$, and $T_3$.
        \EndFor
    \State Compute the proportion of times that the condition in step 4 is satisfied and the average of $T_1, T_2$, and $T_3$.
    \end{algorithmic} % End algorithmic block
\end{algorithm}

Klein et al. (2020)

\section*{Bibliography}\label{bibliography}

\phantomsection\label{refs}
\begin{CSLReferences}{1}{0}
\bibitem[\citeproctext]{ref-klein}
Klein, M., Wright, T., \& Wieczorek, J. (2020). \emph{A joint confidence region for an overall ranking of populations}.

\end{CSLReferences}

\end{document}
